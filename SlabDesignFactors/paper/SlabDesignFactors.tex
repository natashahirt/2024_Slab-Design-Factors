% Preamble
\documentclass[10pt, letterpaper]{article} % Shorter texts using sections and subsections

% Packages
\usepackage[utf8]{inputenc}            % Character encoding
\usepackage{appendix}
\usepackage{algorithm}                 % Algorithm environment
\usepackage{algorithmic}               % Pseudocode formatting
\usepackage{amsmath}                   % Math support
\usepackage{amsfonts}                  % Fonts for math
\usepackage{amssymb}                   % Symbols for math
\usepackage{booktabs}                  % For improving table quality
\usepackage{fancyhdr}                  % For headers and footers
\usepackage{graphicx}                  % For including images
\usepackage{hyperref}                  % For hyperlinks
\usepackage{lipsum}                    % For generating dummy text
\usepackage{listings}                  % For displaying code
\usepackage{subcaption}
\usepackage{titlesec}
\usepackage{tocloft}
\usepackage[backend=biber, sorting=none]{biblatex}
\usepackage[margin=1in]{geometry}      % Adjusted to match letterpaper setting

% Header & footer setup
\pagestyle{fancy}
\fancyhf{}
\fancyfoot[C]{\thepage}                % Center footer - page number

% Paragraph styling 
\setlength{\parskip}{1em}              % Space between paragraphs
\setlength{\parindent}{0em}            % No indent

% Image path
\graphicspath{{assets/}} % Save images in "../assets"

% Bibliography setup
\addbibresource{bibliography/SlabDesignFactors-reference.bib} % Ensure this file exists

% Document properties
\author{Natasha K. Hirt}

% Document
\begin{document}
	
	\title{Topology and design: tools for reducing embodied carbon in steel-and-concrete floor layouts} % Title of the assignment
	\author{Natasha K. Hirt, Caitlin T. Mueller} 
    \date{}      
	\maketitle                              % Generates the title
	
	\section{Motivation and background}

    There exists a worldwide demand for more, and more efficient, buildings. Between 2006 and 2019, global built floor area increased by 65\%, with activity largely concentrated in developing nations. The International Energy Agency estimates that by 2050, the global built floor area will have doubled again.\cite{internationalenergyagencyGlobalBuildingsSector2022} Correspondingly, if conventional practices are upheld, the environmental consequences of the construction industry will increase dramatically. Already, the carbon dioxide equivalent (CO$_2$e) emissions resulting from the extraction, processing, and manufacturing of building materials --- summarily termed embodied carbon (EC) --- account for 7.7\% (2.8GT) of annual global carbon emissions.\cite{architecture2030WhyBuiltEnvironment} To ameliorate the climate impact of future construction, we require new construction methods that allow us to meet the same demand with less material consumption. 

    Flexural systems such as beams and slabs are prime candidates for innovation and improvement. They rank among a building's most material-intensive components, constituting between 40–80\% of the embodied energy of structures up to 80 stories.\cite{foraboschiSustainableStructuralDesign2014,bischofFosteringInnovativeSustainable2022,kromoserRessourceneffizientesBauenMit2020} In part, this owes to the inefficiency of flexural systems compared to elements under axial loading, and in part to the repetition of slabs throughout the structure. Further, the mass of floor slabs has downstream effects on the entire building, impacting the sizes of the columns and foundations required to support them. Lighter floors are therefore a promising strategy for reducing the mass of an entire building.
    
    Lightweight slabs are not a uniquely modern idea. Precedents for efficient slab design have arisen whenever constraints have been imposed the availability of construction materials. For instance, a 1939 ban on steel reinforcement in fascist Italy prompted radical examinations of new structural forms (steel was considered an anti-autarkic material).\cite{ioriPierLuigiNervis1960} Engineer and architect Pier Luigi Nervi thrived in this environment, developing new prefabrication and design methods for concrete and wire-mesh-reinforced ferrocement. His style, recognizable to this day, is characterized by structural slabs and vaults whose ribs run not orthogonally, as is conventional, but flow along the isostatic lines of the structure (Figure 1a). The structural forms were determined using a combination of classical plate theory, strain gauge and photo-elastic experiments, and intuition, ultimately allowing for the economical construction of large spans at a time of great scarcity.\cite{halpernRibbedFloorSlab2013}

    As technology and mathematical knowledge evolved, analytical methods consistently found that better slab design has remarkable material savings potential. Morley’s work in 1966,\cite{morleyMinimumReinforcementConcrete1966} which was further developed by Rozvany and collaborators over the following decades,\cite{rozvanyOptimalDesignFlexural1976} concerns itself with the arrangement of steel reinforcement within concrete slabs so as to achieve optimal moment distributions. Bolbotowski et al. and Whiteley et al. have verified Rozvany’s solutions computationally, and generalized them to accommodate load-dependence.\cite{bolbotowskiDesignOptimumGrillages2018,whiteleyEngineeringDesignOptimized2023} Unfortunately, although the problem is closed-form, is applicable to a range of inital and boundary conditions, and can be generalized with numerical methods, Morley’s initial concern that the “necessary reinforcement arrangement is… too complex for practical use”\cite{morleyMinimumReinforcementConcrete1966} has proven true in standard construction practice.

    Access to powerful computational tools has prompted a revival of interest in optimal floor system design. Additionally, concerns over environmental consequences and EC have underscored the need for practical, readily implementable solutions. In current literature, the two principal approaches to this challenge are shaped beams and optimal beam layouts:

    In a study on shaped beams, Ismail et al. show that carving out underutilized material along the cross-section of a reinforced concrete (RC) beam can reduce its volume by up to 55\%, a result later replicated in timber by the same group.\cite{ismailShapedBeamsUnlocking2021}\cite{mayencourtHybridAnalyticalComputational2020} Both RC and timber can be easily cast, cut, or otherwise shaped to achieve the desired cross-section. Steel, which is the model material utilized in this paper and comprises a large portion of modern-day structural systems, has only been considered a modest candiate for shape optimization, owing both to the already very efficient cross-section of the conventional I-beam, and to the difficulty of variably shaping hot-rolled steel.\cite{leeGeometryStrengthEfficiency2024} Nevertheless, the potential for reducing the material intensity of a beam by leveraging variable shaping is in excess of 40\%. An open-web or castellated beam can reduce material use by 74\%.\cite{carruthTechnicalPotentialReducing2011} 

    The second approach, optimal beam layouts, has several entry points, of which the most salient is continuum topology optimization across an entire slab.\cite{meibodiSmartSlabComputational2018} By algorithmically iterating over a three-dimensional design domain, continuum topology optimization produces a complex pattern of concrete ribs bespoke to the loading and support conditions of that particular slab (Figure 1b).\cite{rippmannDesignFabricationTesting2018} This approach can also be applied to individual beams. \cite{jewettTopologyoptimizedDesignConstruction2019} Despite considerable material savings, the manufacture of nonstandard, vaulted slabs is energy, material, and time-intensive, and a complete life-cycle analysis would be necessary to evaluate their full environmental impact.\cite{liEnergyRequirementsEvaluation2013} An alternative entry point for beam layout optimization is ground-structure topology optimization. Whiteley et al. used this method, which consists of varying the stiffness of members within a dense network until only the essential members remain, to guide the design of reinforcement in a concrete slab.\cite{whiteleyEngineeringDesignOptimized2023} The optimization algorithm used total structural volume as an objective function, and its constraints were compliant with Eurocode 2. Whiteley et al. focus on constructability.Hearkening back to Nervi’s philosophy of economical design, their recommended formwork would be single-sided and reusable for slabs with identical boundary conditions. A subsequent paper then demonstrated the manufacturing process, using laser-cut steel to provide variable levels of reinfrocemnt.\cite{bradburyApplicationsOptimalReinforcement2024} 

    \begin{figure}[ht]
        \centering
        \begin{subfigure}[c]{0.4\textwidth}
            \centering
            \includegraphics[width=\linewidth]{1_nervi_gatti.jpg} 
            \caption{Gatti wool factory by Nervi, 1951. \cite{nerviAestheticsTechnologyBuilding1965}}
            \label{fig:subfigure1}
        \end{subfigure}
        \hfill
        \begin{subfigure}[c]{0.53\textwidth}
            \centering
            \includegraphics[width=\linewidth]{1_block_slab.jpg} 
            \caption{3D sand-printed slab by Block Research Group, 2018. \cite{rippmannDesignFabricationTesting2018}}
            \label{fig:subfigure2}
        \end{subfigure}
        \caption{Historical and modern precedents for shaped slabs that reduce material use. Images by the author.}
        \label{fig:nervi-block}
    \end{figure}

    \subsection{Research gap, questions, and approach}

    Analytical and experimental methods in the literature suggest that optimal flexural system design can significantly reduce the EC of a slab. Extrapolated, this reduces the EC of the entire building. Despite this premise, a generalized, code-compliant design methodology that leverages conventional construction technology has not yet emerged, leading to a growing divide between engineering practice and theoretically derived optimal systems.

    This paper addresses the identified gap by presenting a novel method for computationally analyzing and sizing flat slab geometries with complex beam layouts. The tool allows for a systematic examination of the hypothesis that beam layouts inspired by optimal rib layouts (in vaults) and reinforcement layouts (in unsupported slabs) can reduce the embodied carbon of a flat slab on a range of discretely and parametrically defined layout topologies. It both enables the discovery of new, potentially nonobvious, structural configurations, and, when deployed at scale, illuminates a new conceptual framework for the design of efficient, aesthetically pleasing, and code-compliant floor systems that use commercially available materials and construction methods. 

    The work is divided into two sections. Section~\ref{sec:method} details an algorithmic method for designing slabs with nonstandard steel beam layouts. The slabs considered are flat reinforced concrete plates and supported by wide-flange (W) steel sections ($E = 29000$ ksi). The method first describes tributary area analysis and load assignment in arbitrary geometries for isotropic, uniaxial, and biaxial orthogonal slabs. Subsequently, it covers the sizing of beams using discrete (from a catalogue) and continuous optimization.

    Section~\ref{sec:results} shows how the method can be applied to estimate the effect of detailing and topology decisions on the overall mass and EC of the slab. Both manually and parametrically defined design spaces are considered.

    The primary research questions guiding this work are:

    \begin{itemize}
        \item How can we effectively analyze the performance of beam-slab structural systems with nonstandard beam layouts?
        \item What are the most impactful drivers of efficient floor design (Section~\ref{sec:design-impacts})? Qualitatively, what are the trade-offs with manufacturability (Section~\ref{sec:moment-releases})?
        \item How sensitive are parametrically-defined topologies to changes in beam layout density and other parameters (Section~\ref{sec:parametric-topologies})?
    \end{itemize}

    This work employs four of Fang et al.’s thirteen design strategies for reducing structural EC. (\textit{0a.} Estimating EC from bottom-up quantities; \textit{1.} Exploring or optimizing the parametric design space; \textit{3.} Using less material; \textit{9.} Exploiting standardization and/or customization).\cite{demifangReducingEmbodiedCarbon2023}

    \section{Method: Automated embodied carbon assessment of beam-slab structural systems}\label{sec:method}

    \subsection{Conceptual overview}

    \subsection{Design and detailing decisions}\label{sec:design-decisions}

    \subsubsection{Slab type: tributary areas and load distribution}

    \textit{Isotropic slabs}

    \textit{Uniaxial slabs}

    \textit{Biaxial orthogonal slabs}

    \subsubsection{Slab sizing: maximum span and slab depth}

    slab depth for isotropic + orth-biaxial: ACI 318-19, Table 8.3.1.1
    
    - interior panels without drop panels, conservatively using the value for fy = 280 MPa (i.e. reinforcement grade 280) because the unusual geometries might lead to unexpected slab behaviour: L/36

    - minimum thickness is also kept conservative at 125mm (5"), from table 8.3.1.2

    slab depth for uniaxial: ACI Code Table 9.5a
    
    - both ends continuous: L/28

    \subsubsection{Beam sizing: discrete and continuous optimization}

    \subsubsection{Beam collinearity: identifying same-sized beams}

    \subsubsection{Assembly depth: total assembly depth and beam depth}

    \subsection{Layout topology}

    \subsection{Summary of method}

    \subsubsection{Business-as-usual (BaU)}

    \section{Results}\label{sec:results}

    \subsection{Impact of design and detailing decisions}\label{sec:design-impacts}

    \subsubsection{Slab sizing}

    \subsubsection{Beam sizing}

    \subsubsection{Beam collinearity}

    \subsubsection{Assembly depth}

    \subsubsection{Slab type}

    \subsubsection{Slab typologies}

    \subsection{Impact of topology variation}

    \subsubsection{Discrete topologies}

    \subsubsection{Parametric topologies}\label{sec:parametric-topologies}

    \subsection{Practical considerations: moment releases}\label{sec:moment-releases}

    \subsection{Discussion}

    \section{Conclusions}

    \subsection{Summary of contributions and findings}

    \subsection{Potential impacts}

    \subsection{Limitations and future work}

    Talk about Ramon's paper, cite him, algorithmically generating layouts given a slab perimeter geometry

    \subsection{Concluding remarks}

    \begin{appendices}
        \section{Results tables}
        \subsection{Full dataset}
        \subsection{Slabs better than business-as-usual}
        \section{JSON sample structure}
    \end{appendices}

    % Add bibliography
    \printbibliography

\end{document}
